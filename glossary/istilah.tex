\newglossaryentry{atribut} {
	name        = atribut,
	description = {ukuran yang digunakan untuk merepresentasikan sebuah \textit{instance}},
	sort        = 2atribut
}

\newglossaryentry{fitur} {
	name        = fitur,
	description = {aspek yang dilihat untuk membedakan satu \textit{instance} dari yang lain},
	sort        = 2fitur
}

\newglossaryentry{hipotesis} {
	name        = hipotesis,
	description = {sebuah fungsi yang berusaha mengestimasi konsep/fungsi target sedekat-dekatnya},
	sort        = 2hipotesis
}

\newglossaryentry{instance} {
	name        = \textit{instance},
	description = {satuan individu/anggota dari domain yang dipelajari},
	plural      = \textit{instances},
	sort        = 2instance
}

\newglossaryentry{konsep} {
	name        = konsep,
	description = {sebuah fungsi \textit{boolean} yang menyatakan apakah sebuah \textit{instance} masuk ke dalam suatu kelas atau tidak},
	sort        = 2konsep
}

\newglossaryentry{korpus} {
	name        = korpus,
	description = {sekumpulan sumber teks atau suara yang dapat dibaca oleh mesin yang ditujukan untuk penelitian pemrosesan bahasa},
	plural      = korpora,
	sort        = 2korpus
}

\newglossaryentry{ngram} {
	name        = \textit{$n$-gram},
	description = {barisan $n$ kata berurutan dalam sebuah teks},
	sort        = 2ngram
}

\newglossaryentry{stopword} {
	name        = \textit{stop word},
	description = {kata yang sering muncul dalam dokumen, tetapi tidak ada kaitan erat dengan topik yang dibahas seperti kata hubung dan kata sandang},
	sort        = 2stopword
}