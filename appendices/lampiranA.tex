\documentclass[../sections/lampiran.tex]{subfiles}

\begin{document}
\newappendix{Posisi Tugas Akhir}

\newsubappendix{Tabel Posisi}
Pembangunan kamus dwibahasa termasuk salah satu penerapan dari ekstraksi istilah dwibahasa. Posisi tugas akhir dalam studi literatur diberikan pada Tabel \ref{a:posisi_tugas_akhir}.
\begin{table}[htbp]
	\centering
	\caption{Posisi tugas akhir dalam studi literatur}
	\label{a:posisi_tugas_akhir}
	\begin{tabular}{p{\tablewidth/6} p{\tablewidth/10} p{\tablewidth/10} p{\tablewidth/10} p{\tablewidth/10} p{\tablewidth/10} p{\tablewidth/10} p{\tablewidth/8}}
		\hline
		\multirow{3}{*}{\centering\textbf{Studi}} & \multicolumn{6}{c}{\centering\textbf{ATR}} & \multirow{3}{*}{\centering\textbf{Bahasa}}\\
		\cline{2-7}
		{} & \multicolumn{2}{p{\tablewidth/5}}{\centering banyak bahasa} & \multicolumn{2}{p{\tablewidth/5}}{\centering pendekatan} & \multicolumn{2}{p{\tablewidth/5}}{\centering korpus} & {}\\
		\cline{2-7}
		{} & \centering eka & \centering dwi & \centering ling. & \centering stat. & \centering par. & \centering \textit{comp.} & {}\\
		\hline
		\hline
		\textcite{ananiadou} & V & {} & V & {} & V & V & Inggris\\ \hline
		\textcite{tsuji}     & {} & V & V & {} & V & V & Jepang-Prancis\\ \hline
		\textcite{fujii}     & {} & V & V & {} & V & V & Jepang-Korea\\ \hline
		\textcite{daille}    & {} & V & {} & V & {} & V & Inggris-Prancis\\ \hline
		\textcite{lefever}   & {} & V & {} & V & V & {} & Belanda-Inggris-Italia-Prancis\\ \hline
		\textcite{aker}      & {} & V & V & V & {} & V & 22 negara Eropa\\ \hline
		\textcite{limanthie} & {} & V & {} & V & {} & V & Indonesia-Jepang\\
		\hline
		\hline
		\textcite{tugas akhir} & \centering {} & V & V & V & {} & V & Indonesia-Jepang\\
		\hline
	\end{tabular}
\end{table}

\newsubappendix{Anak Lampiran Lainnya}
appendix  appendix appendix appendix appendix appendix
\end{document}